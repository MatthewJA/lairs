%!tex root=./thesis.tex

\chapter*{List of Constants}
\addcontentsline{toc}{chapter}{List of Constants}
\markboth{List of Constants}{}

The values of the following constants, except where otherwise noted, are drawn from the NIST Reference on Constants, Units, and Uncertainty \citep{mohr_nist_2019} which itself draws from the 2018 CODATA recommended values.

\begin{longtable}{llll}
\toprule
\bfseries Symbol & \bfseries Unit & \bfseries Name & \bfseries Value \\\midrule\endhead
$\epsilon_0$ & F m$^{-1}$ & Vacuum permittivity & $8.8541878128(13) \times 10^{-12}$ \\
$G$ & m$^3$ kg$^{-1}$ s$^{-2}$ & Gravitational constant & $6.67430(15) \times 10^{-11}$ \\
$m_p$ & kg & Proton mass & $1.67262192369(51) \times 10^{-27}$ \\
$m_e$ & kg & Electron mass & $9.1093837015(28) \times 10^{-31}$\\
$c$ & m s$^{-1}$ & Speed of light & $2.99792458 \times 10^8$ \\
$\sigma_T$ & m$^2$ & Thomson cross section & $6.6524587321(60) \times 10^{-29}$\\
\bottomrule
\end{longtable}

\chapter*{List of Abbreviations}
\addcontentsline{toc}{chapter}{List of Abbreviations}
\markboth{List of Abbreviations}{}

The following list summarises abbreviations that are commonly used in this thesis.

\begin{itemize}
    \item AGN: active galactic nuclei, energetic objects at the centre of galaxies
    \item ASKAP: Australian Square Kilometre Array Pathfinder, a next-generation radio telescope in Murchison
    \item ATCA: Australia Telescope Compact Array, a radio telescope in Narrabri
    \item CNN: convolutional neural network, a classifier which works on images and spectra
    \item EMU: Evolutionary Map of the Universe, an upcoming large radio survey
    \item FDF: Faraday dispersion function, a representation of a polarisation spectrum
    \item FIRST: Faint Images of the Radio Sky at Twenty Centimeters, a large radio survey
    \item FRI: Fanaroff-Riley type I, an edge-darkened radio galaxy
    \item FRII: Fanaroff-Riley type II, an edge-brightned radio galaxy
    \item ISM: interstellar medium, the stuff between stars
    \item LR: logistic regression, a classification model
    \item MWA: Murchison Widefield Array, a next-generation radio telescope in Murchison
    \item NVSS: NRAO VLA Sky Survey, a large radio survey
    \item POSSUM: Polarisation Sky Survey of the Universe's Magnetism, an upcoming large radio polarisation survey
    \item RACS: Rapid ASKAP Continuum Survey, a new large radio survey
    \item RF: random forests, a classification model
    \item RGZ: Radio Galaxy Zoo, a citizen science project to cross-identify and aggregate radio sources
    \item RLF: radio luminosity function, a description of how common radio galaxies of different energies are
    \item RM: rotation measure, the amount of Faraday rotation between a polarised source and an observer
    \item RMSF: rotation measure spread function, the kernel convolving a FDF
    \item SDSS: Sloan Digital Sky Survey, a large optical spectroscopic and photometric survey
    \item SFR: star formation rate, the recent rate of star formation
    \item SKA: Square Kilometre Array, a next-generation radio telescope yet to be built
    \item SNR: signal-to-noise ratio, the ratio of total intensity to noise level
    \item SWIRE: \emph{Spitzer} Wide-area Infrared Extragalactic Survey, a deep infrared survey
    \item VLA: Very Large Array, a radio telescope in New Mexico
    \item XGB: extreme gradient boosted trees, a classification model
\end{itemize}