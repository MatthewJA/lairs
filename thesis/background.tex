%!tex root=./thesis.tex
\chapter{Background}
\label{cha:background}

I'll need to explain the purpose of this chapter and give a brief outline. There'll be three sections, and I'll briefly summarise them here.

\section{Astrophysics of Extended Radio Sources}
\label{sec:physics-of-radio-sources}

In this section I'll talk about extended radio sources: AGN, their jets, and the unified model as described by Urry and Padovani. I'll need to talk about synchrotron radiation and the AGN core and jets. It'd also make sense to talk about observational classes, including FRI and FRII, and how orientation affects this. How do radio objects evolve over time? What are the main questions in the field?


\section{Observing Radio Sources}
\label{sec:radio-astronomy}

This section needs to talk about how the physics of AGN affects observations and what kind of data we deal with. I'll need to talk about radio telescopes, Fourier transforms, and the noise properties of radio sources, as well as what kind of sources we expect to find throughout the universe. How do observations limit our understanding of AGN? How do observational effects change what we see? What are the main difficulties in radio astronomy?


\section{Machine Learning for Radio Astroninformatics}
\label{sec:radio-astroinformatics}

Here we need to talk about the fundamentals of machine learning. We'll start with problem formulations and what machine learning is, then discuss terminology and classification. We need to cover the difficulties of labels and the lack of groundtruth in astronomy, the problem and handling of uncertainties, and feature selection. We also need to talk about the importance (or unimportance?) of interpretability and how this ties into astronomy, and the unique idea of using machine learning as a pathway to understand something important about physics.
