%!tex engine=xelatex
%!tex root=./thesis.tex
\chapter{Conclusion}
\label{cha:conc}

This chapter needs to
\begin{itemize}
    \item summarise the thesis
    \item draw together works
    \item link each chapter to theory and methodology
    \item evaluate the contribution of this work to the field
    \item discuss future work
\end{itemize}

The future of radio astronomy is immensely exciting, with upcoming radio surveys like EMU and POSSUM sure to revolutionise our understanding of the violent Universe occupied by radio active galactic nuclei. In this thesis we have explored methods for identifying extragalactic radio sources in these future wide-area radio surveys.

\autoref{cha:cross-id} developed a new approach to cross-identifying radio components with their infrared host galaxies and applied this method to the CDFS and ELAIS-S1 ATLAS survey, a pilot survey for the upcoming ASKAP key survey EMU. This was the first ever application of machine learning to radio cross-identification. In the process, we demonstrated that our methods worked on ELAIS-S1 with similar performance to how they worked on CDFS, despite these being different parts of the sky. We also evaluated simple nearest-neighbour cross-identification on the ATLAS fields, showing better performance than either machine learning or Radio Galaxy Zoo volunteers on this dataset, which we suggest is because of how most objects in ATLAS are very compact. Our machine learning methods performed similarly whether they were trained on expert-provided or citizen science-provided labelled datasets, clearly demonstrating the benefit of citizen science labels to radio astroinformatics even if they may be less accurate than labels provided by expert. We showed that the algorithm we had developed could in principle perform much better than nearest-neighbours if only its binary accuracy could be raised, perhaps with a more complex or better-trained classifier. Finally, we concluded that ATLAS was an insufficiently complex dataset to train machine learning algorithms for future radio surveys: more complex and larger training sets would be necessary.

To further investigate our classification-based cross-identification algorithm we needed a more complex dataset, and in \autoref{cha:rlfs} we turned to FIRST, a legacy survey containing around 900~000 radio components. Of these, 250~000 or so were not compact, compared to the vast majority of ATLAS. We increased the complexity of our machine learning model and trained it on this dataset along with 75~000 Radio Galaxy Zoo labels, far more than we had available for ATLAS. The resulting classifier could then be used to cross-identify every extended object in FIRST. Using the fact that any two radio components in the same radio source should also have the same host galaxy, we identified not only the host galaxies of our radio components but also their association to other radio components. This allowed us to produce a catalogue of over 150~000 extended radio sources, the largest extant catalogue of extended radio sources at the time of writing. Such a large catalogue enabled us to estimate a RLF for extended radio sources, the first time a RLF has been produced for just extended sources. We could also subdivide this RLF into a fractional RLF, and we were able to produce RLFs divided by the mid-infrared colour of the host galaxy, their physical extent, and their redshift. Extended radio sources ought to be the sources which contribute the most mechanical energy to their surrounding IGM, and with an RLF dedicated solely to extended sources we were able to estimate this energy contribution as $1.3 \times 10^{30}$ and $1.2 \times 10^{32}$~W~Mpc$^{-3}$. Perhaps most importantly, we showed our method works and used it to obtain a physically meaningful result. Our method can easily be extended to new surveys, as long as training data exist.

In \autoref{cha:faraday} we developed an interpretable Faraday complexity classifier. We constructed features that were easy to understand by measuring the distance of observations from a simple model of Faraday simple sources. Our resulting features could be calculated from both simulated and real observations. We demonstrated the effectiveness of our method on both simulated and real data and showed that on simulated data our simple methods matched the state-of-the-art CNN classifier despite having far less parameters. This was the first application of machine learning to real spectropolarimetric data. This method will be applicable to future surveys like POSSUM.

\section{Implications for radio citizen science}
\label{sec:implications-citizen-science}

    Our results in \autoref{cha:cross-id} demonstrated that machine learning methods trained on citizen science labels perform comparably to those trained on expert labels, even when those labels are lower quality than expert labels. We applied this insight in \autoref{cha:rlfs} to obtain scientific results, using machine learning to extrapolate the labels to a larger dataset. While Radio Galaxy Zoo alone was not enough to fully label FIRST, Radio Galaxy Zoo working in tandem with machine learning was. This is a pattern that may hold true for future surveys and applications, too. Future astronomical research at-scale may leverage the idea of people working with machine learning, sometimes called \defn{human-in-the-loop} \citep[e.g.][]{holzinger_interactive_2016} learning, to pore through data-at-scale.

\section{Implications for wide-area radio surveys}
\label{sec:implications-wide-area-radio-surveys}

    As we move toward larger and larger datasets, an important question is how applicable our models will be across the sky. Our results in \autoref{cha:cross-id} showed that we can expect some generalisation, as our model trained on one patch of sky was applicable to another without a great loss of performance. Similarly, our classifier trained on part of FIRST seemed to work well on the rest of the dataset. This is promising as it implies that limited area surveys may provide 

    Between \autoref{cha:cross-id} and \autoref{cha:rlfs} we demonstrated that a large set of good quality, complex data is required for training good astroinformatics algorithms. Pilot datasets like ATLAS will not work by themselves: the sources they contain are too simple and their complex sources are too few. A sensible question to ask is, could we simulate data for training purposes? We trained our classifier in \autoref{cha:faraday} on simulated data and found that it was difficult to bridge the domain gap between simulation and observation, even in a well-understood, one-dimensional case---let alone the complex three-dimensional projected morphologies we observe in imagery. Getting across this domain gap will be difficult and will necessarily be a major topic of research in the astroinformatics field in the near future.

    Tying observations, simulations, and models together are the representation of the data: features. Our results in \autoref{cha:faraday} show that judicious choice of features 

    This section discusses the links between my work and upcoming wide-area radio surveys, including any predictions I may have for them, and impacts of my work on them.
    - good generalisation across the sky - from cdfs/elais and first
    - good, complex data are necessary for training
    - we need to pick features cleverly to make the best use of the limited radio sky - we only have one, and can't perform repeated experiments like other fields!

\section{What we need to move forward}
\label{sec:moving-forward}
    
    What does radio astronomy need to be able to press ahead with astroinformatics? From my work what do we need to do to set ourselves up for a solid future field?

\section{Future Work}
\label{sec:future}

    Where can my work be taken from here?

    - domain gap for training on simulated data for FDFs
    - domain gap for simulated radio continuum data to augment surveys like ATLAS
    - apply our cross-identification and Faraday complexity methods to EMU and POSSUM, as well as surveys like LoTSS and GLEAM-X
    - improve our aggregation method for bxid
    - generate alternative fractional RLFs, e.g. by ML-derived morphology
    - incorporate photometric redshifts into calculations from rlf
    - investigate the link between extended sources and their local environments
    - apply other crowd aggregation strategies to RGZ
    - use our insights into FDFs to do other interesting things, like outlier detection, data visualisation, or some new way of qu fitting
